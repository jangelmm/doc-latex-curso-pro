\section{Ecuaciones}
Sabemos que $2+2=4$, lo cual es una suma. Pero en el caso de la multiplicación seria algo como:

$$ 2\cdot 2 = 4 $$

Para definir un entorno de ecuaciones utilizamos lo siguiente

\begin{equation}
    \int_{a}^{b} f(x)\cdot dx = 2x^2+\sin(x)
    \label{eqn:ecuacion1}
\end{equation}

\begin{equation*}
    \oint_{a}^{b} f(x)\cdot dx = 2x^2+\sin(x)
    \label{eqn:ecuacion2}
\end{equation*}

Si quiero citar la ecuacion anterior, lo hago con ref, y seria por ejemplo, la ecuacion \ref{eqn:ecuacion1}

\begin{align}
    %f(x) = 2x^3+\cos(x) + \ln(bx+3) +log(x^7) + \alpha\int_a^b g dx
    f(x) &= 2x^3+\cos(x) \\
         & \hspace{10pt} + \ln(bx+3) +log(x^7) \\ 
         & \hspace{10pt} + \alpha\int_a^b g dx 
\end{align}

Lo que utilizo para enumerar estas ecuaciones divididas

\begin{equation}
    \begin{split}
        f(x) & = 2x^3+\cos(x) \\
             & \hspace{10pt} + \ln(bx+3) +log(x^7) \\ 
            & \hspace{10pt} + \alpha\int_a^b g dx 
    \end{split}
    \label{eqn:ecuacion3}
\end{equation}

Para alinear ecuaciones en el centro

\begin{gather} %Recomendado en sistemas de ecuaciones
    f(x) = 2x^3+\cos(x) \\
         \hspace{10pt} + \ln(bx+3) +log(x^7) \\ 
         \hspace{10pt} + \alpha\int_a^b g dx 
\end{gather}

\begin{gather} %Recomendado en sistemas de ecuaciones
    2x+3=5 \\
    \sin(x) + x^4 = 3
\end{gather}

Para ingresar texto entremedio del entorno equation:

\begin{equation}
    \begin{bmatrix}
        a & b \\
        c & d
    \end{bmatrix}
    =
    \begin{bmatrix}
        1 \\
        2
    \end{bmatrix}
    \cdot
    \begin{bmatrix}
        3 & 4
    \end{bmatrix}
    \overset{\text{implicancia}}{\rightarrow}
    \lim_{x \to 2} f(x) in \mathbb{R}
\end{equation}

Algunas fuentes matemáticas son: $\mathbb{R}$, $\mathscr{R}$, $\mathfrak{R}$, $\mathcal{R}$, $\mathtt{R}$.

\newpage

Para definir una unidad a la mala podemos hacer;
\begin{equation}
    2 \ \text{mm}
\end{equation}

Para una mejor forma es utilizar el paquete siunitx

\begin{gather}
    2 \ \unit{kg.mol^{-1}} \\
    2 \ \unit{\kilogram\per\mole}
    \qty{2}{kg.mol^{-1}} %Recomendada
\end{gather}